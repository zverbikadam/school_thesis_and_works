
\chapter*{Úvod}
\addcontentsline{toc}{chapter}{Úvod}

Internet. Dnes si život bez neho vieme iba ťažko predstaviť. Obrovské množstvo ľudí sa pripojí na internet každý jeden deň. Či už si chcú nájsť nejakú informáciu, objednať niečo z internetového obchodu, pozrieť seriál, film, alebo chcú komunikovať s ďalšími ľuďmi, pripojenými na druhom konci sveta. Internet je stále dostupnejší. Vďaka novým, vyspelejším technológiám sa stáva pripojenie stabilnejšie a rýchlejšie. Práve kvôli tomu sú internetové aplikácie stále populárnejšie. Takáto aplikácia je plnohodnotná aplikácia, ale používateľ si nemusí nič inštalovať do svojho počítača, jednoducho sa pripojí na internet a v pohodlí internetového prehliadača si aplikáciu spustí.

Vzhľadom na neustály pokrok technológií sú stále výkonnejšie a dostupnejšie aj samotné počítače. S rastúcim výkonom počítačov rastie aj popularita jazyka JavaScript a jeho frameworkov, nakoľko moderné počítače si s ním hravo poradia. To umožňuje vývoj moderných, dynamických aplikácií s použitím internetových technológií. Takýmito frameworkami sú aj React a Electron. React umožňuje beh jedno-oknovej (single-page) aplikácie v internetovom prehliadači, Electron zase v okne operačného systému. Kombináciou týchto dvoch frameworkov vznikne multiplatformná desktopová aplikácia dostupná aj z internetového prehliadača.

Senzorický informačný systém Sensorical je systém, ktorý zaznamenáva a analyzuje namerané environmentálne dáta. Dáta sú zaznamenávané meracími zariadeniami, ktoré merajú hodnoty v určitých časových intervaloch a po zaznamenaní tejto hodnoty komunikujú so vzdialeným serverom, kde sa tieto dáta ukladajú. Tento informačný systém obsahuje desktopovú aplikáciu, internetovú aplikáciu a aplikáciu pre operačný systém Android.  

Cieľom práce je naprogramovať a implementovať multiplatformnú desktopovú aplikáciu, ktorá dostala názov eAurela. Aplikácia eAurela bude dostupná aj z internetového prehliadača, čo značne zjednoduší spravovanie celého systému Sensorical. Aplikácia bude zobrazovať posledné namerané hodnoty z meracích zariadení a grafické výstupy z vybraných časových intervalov. Administrátori informačného systému budú mať možnosť z aplikácie spravovať celý systém.

\begin{comment}

\begin{code}
\begin{minted}{js}

\end{minted}
\end{code}
\end{comment}