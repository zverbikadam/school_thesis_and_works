\fancyhead[L, R]{MTF STU \hfill BAKALÁRSKA PRÁCA }

%--------------------------------------------------------------------------------------
%%% slovensky abstrakt
\noindent{\textbf{Súhrn}}


\noindent
Zverbík, Adam: Klientská aplikácia pre zobrazenie dát ako súčasť senzorického informačného systému. [Bakalárska práca]. – Slovenská technická univerzita. Materiálovotechnologická fakulta; Ústav aplikovanej informatiky automatizácie a mechatroniky. – Vedúci práce: Ing. Juraj Ďuďák, PhD. – MTF STU, 2020, 57s.

\bigskip
Cieľom záverečnej práce je vyvinúť a implementovať multiplatformnú desktopovú aplikáciu a web aplikáciu. Výsledkom tejto práce je funkčná klientská aplikácia pre zobrazenie dát a spravovanie senzorického informačného systému Sensorical. V prvej kapitole je opísaný systém Sensorical, jeho dátový model a komunikácia so serverom. Druhá kapitola sa venuje analýze požiadaviek a štruktúre aplikácie. V tretej kapitole sú popísané nástroje, ktoré boli použité pri vývoji aplikácie. Štvrtá kapitola pojednáva o implementácií a piata kapitola je venovaná buildovaniu aplikácie.
\bigskip

\noindent
\textbf{Kľúčové slová}:  desktopová aplikácia, web aplikácia, React, Electron, senzorický informačný systém
\bigskip


\bigskip
\noindent
\textbf{Abstract}

\noindent 
Zverbík, Adam: Client application for data visualization as a part of sensory information system. [Bachelor thesis]. – Slovak University of Technology. Faculty of Materials Science and Technology; Institute of Applied Informatics, Automation and Mechatronics. – Thesis supervisor: Ing. Juraj Ďuďák, PhD. – MTF STU, 2020, 57p.


\bigskip
The aim of the bachelor thesis is to develop and implement multiplatform desktop application and web application. The result of this thesis is functional client application, which shows and manages data of the sensoric information system Sensorical. Data model of sensorical system and communication with server is described in the first chaper. The second chapter is dedicated to analyze requirements and structure of the application. The tools used in development process are described in the third chapter. The fourth chapter is about implementation and the fifth chapter describes building of the final application.

\bigskip
\noindent
\textbf{Key words: } desktop application, web application, React, Electron, sensory information system. 



