%hidelinks- zrusi cervene oramovanie v pdf
\documentclass[a4paper,12pt,hidelinks]{report}
\usepackage[slovak,english]{babel}
\usepackage[utf8]{inputenc}  
\usepackage[protrusion=true,expansion=true]{microtype}
%\usepackage{mathpazo}
\usepackage{mathptmx}
\usepackage[T1]{fontenc}  
\usepackage[table,xcdraw]{xcolor}
\linespread{1.2} %1.2
%alebo cez \renewcommand{\baselinestretch}{1.5}\normalsize
\usepackage{moresize}
%\usepackage{enumitem}
%\usepackage{geometry}
\usepackage[a4paper,left=3.2cm,right=2.2cm,top=2.5cm,bottom=3cm]{geometry}
%\addtolength{\oddsidemargin}{-.4cm}
%\addtolength{\evensidemargin}{-3.2cm}
%\addtolength{\topmargin}{-1cm}
%\addtolength{\textheight}{0cm}
%\addtolength{\textwidth}{1.4cm}
%\addtolength{\textheight}{1.4cm}
\usepackage{graphicx}
\usepackage{float}
\usepackage{subfig}
\usepackage{pdfpages}
\usepackage{listings}
\usepackage{eurosym}
\usepackage{url,mathptmx}
\usepackage{booktabs}
\usepackage{array, hhline}
\usepackage{lscape}
%\usepackage{amssymb, amsmath, amsfonts, lmodern}
\usepackage{amssymb, amsmath, lmodern}
%toto je trochu ine pismo \usepackage{amssymb,amsfonts,amscd}
\usepackage{acronym}
\usepackage{paralist}
%\usepackage{verbatim}
\usepackage{minted} %minted
%\usemintedstyle{colorful} %minted
%==============zrusenie cerveneho ramiku pri chybe
\makeatletter
\AtBeginEnvironment{minted}{\dontdofcolorbox}
\def\dontdofcolorbox{\renewcommand\fcolorbox[4][]{##4}}
\makeatother
%==============zrusenie cerveneho ramiku pri chybe
\setminted[json]{fontsize=\footnotesize} %minted
\setminted[java]{fontsize=\footnotesize} %minted
\setminted[js]{fontsize=\footnotesize} %minted
\usepackage{chngcntr} %minted
\counterwithin{listing}{chapter} %minted

\newenvironment{code}{\captionsetup{type=listing}}{}

\usepackage{comment}
\usepackage[colorinlistoftodos]{todonotes}
\usepackage[pdftex,unicode,bookmarks=false]{hyperref}
\usepackage{fancyhdr}
\usepackage{csquotes}
%\usepackage{subcaption} %package pre moznost mat 2 obrazky vedla seba
\usepackage{nameref} %package pre moznost referencovania celym menom kapitol

%\tolerance=1
%\emergencystretch=\maxdimen
%\hyphenpenalty=10000
%\hbadness=10000

\renewcommand{\chaptermark}[1] % HEADER / FOOTER  DEFINITIONS
{
  \markboth{#1}{}
}
\usepackage{sectsty}% SECTION TITLE APPEARANCE
\allsectionsfont{\mdseries\upshape}
\usepackage[explicit]{titlesec}

\usepackage[nottoc,notlof,notlot]{tocbibind}
\usepackage[titles,subfigure]{tocloft}
\renewcommand{\cftsecfont}{\rmfamily\mdseries\upshape}
\renewcommand{\cftsecpagefont} {\rmfamily\mdseries\upshape}

\usepackage{nicefrac}
\usepackage[backend=biber,style=numeric,sorting=none]{biblatex}

\definecolor{lightgray}{rgb}{.9,.9,.9}
\definecolor{darkgray}{rgb}{.4,.4,.4}
\definecolor{purple}{rgb}{0.65, 0.12, 0.82}
\definecolor{darkgreen}{rgb}{0.0, 0.5, 0.0}
\definecolor{denim}{rgb}{0.08, 0.38, 0.74}
\definecolor{carrotorange}{rgb}{0.93, 0.57, 0.13}
\definecolor{darkblue}{rgb}{0.0, 0.0, 0.55}

\lstdefinelanguage{JavaScript}{
  keywords={typeof, new, true, false, catch, function, return, null, catch, switch, var, if, in, while, do, else, case, break, try},
  keywords=[3]{fetch, then, setState,log, alert, reload, json},
  keywords=[4]{const, let, props, state, this},
  keywordstyle=\color{red}\bfseries,
  keywordstyle=[3]\color{black},
  keywordstyle=[4]\color{darkblue},
  ndkeywords={class, export, boolean, throw, implements, import, from},
  ndkeywordstyle=\color{purple}\bfseries,
  identifierstyle=\color{black},
  sensitive=false,
  comment=[l]{//},
  morecomment=[s]{/*}{*/},
  commentstyle=\color{darkgreen}\ttfamily,
  stringstyle=\color{carrotorange}\ttfamily,
  morestring=[b]',
  morestring=[b]"
}
\lstset{
literate=%
         {á}{{\'a}}1
         {í}{{\'i}}1
         {é}{{\'e}}1
         {ý}{{\'y}}1
         {ú}{{\'u}}1
         {ó}{{\'o}}1
         {ě}{{\v{e}}}1
         {š}{{\v{s}}}1
         {č}{{\v{c}}}1
         {ř}{{\v{r}}}1
         {ž}{{\v{z}}}1
         {ď}{{\v{d}}}1
         {ť}{{\v{t}}}1
         {ľ}{{\v{l}}}1
         {ň}{{\v{n}}}1                
         {ů}{{\r{u}}}1
         {Á}{{\'A}}1
         {Í}{{\'I}}1
         {É}{{\'E}}1
         {Ý}{{\'Y}}1
         {Ú}{{\'U}}1
         {Ó}{{\'O}}1
         {Ě}{{\v{E}}}1
         {Š}{{\v{S}}}1
         {Č}{{\v{C}}}1
         {Ř}{{\v{R}}}1
         {Ž}{{\v{Z}}}1
         {Ď}{{\v{D}}}1
         {Ť}{{\v{T}}}1
         {Ň}{{\v{N}}}1                
         {Ů}{{\r{U}}}1 
         {ô}{{\^{o}}}1
}
\lstset{
   language=JavaScript,
   backgroundcolor=\color{white},
   extendedchars=true,
   basicstyle=\footnotesize\ttfamily,
   showstringspaces=false,
   showspaces=false,
   %numbers=left,
   numberstyle=\footnotesize,
   numbersep=9pt,
   tabsize=2,
   breaklines=true,
   showtabs=false,
   captionpos=b
}
%\lstset{style=mystyle}

\addbibresource{literatura.bib}

\newtheorem{mydef}{Definícia}
\newtheorem{myexample}{Príklad}

\newcommand\sbref[1]{[\ref{#1}]}
\newcommand\cref[1]{Kód \ref{#1}}
\newcommand\picref[1]{Obrázok \ref{#1}}

\fancypagestyle{plain}{%
  \fancyhf{}%
  %\fancyhead[L, R]{MTF STU \hfill BAKALÁRSKA PRÁCA } hlavicka MTF STU
   \fancyfoot[C]{}
   \fancyfoot[R]{\thepage}
}
 \renewcommand{\headrulewidth}{0pt} % remove lines as well

\titlespacing*{\chapter}{0pt}{-30pt}{30pt}

\titleformat{\section}{\Large\bf}{\textcolor{black}{\thesection \hspace{12pt} #1}}{0pt}{}
%\titleformat{\subsection}{\large\bf}{\textcolor{black}{#1}}{0pt}{}
\subsubsectionfont{\bf}

%pridane
\subsectionfont{\bf}
%pridane
%\usepackage[font=small]{caption} %font pre caption v obrazkoch

%zmena nazvu caption Listing
\renewcommand\listingscaption{Kód} %minted
\renewcommand\lstlistingname{Kód}
\renewcommand\lstlistlistingname{Zdrojové kódy}

\begin{document}
\selectlanguage{slovak}

\titleformat{\chapter}[display]{\normalfont\bfseries}{}{10pt}{\huge#1}
%\titleformat{\chapter}[display]{\normalfont\bfseries}{}{10pt}{\huge\thechapter \quad#1}


\pagenumbering{gobble}  %roman

\pagestyle{empty}
\include{special/obal.tex}
\includepdf{prilohy/obal.pdf}
\includepdf{prilohy/title.pdf}
\includepdf{prilohy/zadanie.pdf}
%naskenovat a pridat do priecinka
%\includepdf{zadanie.pdf}   %toto odkomentovat pri tlaceni, alebo ked tam budete mat naskenovane zadanie, alebo zadanie v pdf


\pagestyle{plain}
\setcounter{page}{3}

\newpage
\phantom.
\vspace{12cm}
\noindent{\bf Čestné vyhlásenie}

\vspace{2em}

Prehlasujem, že som XXXXXXXX prácu \textit{Nazov tvojej prace} vypracoval/a samostatne pod vedením ... , a uviedl/a v nej všetky použité literárne a iné odborné zdroje v súlade s právnymi predpismi, vnútornými predpismi Žilinskej univerzity a vnútornými aktmi riadenia Žilinskej univerzity a Fakulty riadenia a informatiky.

\vspace{2em}

\noindent
V~Žiline, dňa ..............
\hfill
..........................................


\hfill
Meno Priezvisko \qquad\quad

\newpage
\phantom.
\vspace{12cm}
\noindent{\bf Poďakovanie}
\vspace{2em}
\noindent
\bigskip

Na tomto mieste by som chcela poďakovať vedúcemu diplomovej práce....  za cenné pripomienky a odborné rady, ktorými prispel k vypracovaniu tejto XXXXX práce. Taktiež ďakujem môjmu ... . Zároveň ďakujem mojej rodine a priateľom za ich nekonečnú podporu a trpezlivosť. 

\vspace{2em}


\fancyhead[L, R]{MTF STU \hfill BAKALÁRSKA PRÁCA }

%--------------------------------------------------------------------------------------
%%% slovensky abstrakt
\noindent{\textbf{Súhrn}}


\noindent
Zverbík, Adam: Klientská aplikácia pre zobrazenie dát ako súčasť senzorického informačného systému. [Bakalárska práca]. – Slovenská technická univerzita. Materiálovotechnologická fakulta; Ústav aplikovanej informatiky automatizácie a mechatroniky. – Vedúci práce: Ing. Juraj Ďuďák, PhD. – MTF STU, 2020, 57s.

\bigskip
Cieľom záverečnej práce je vyvinúť a implementovať multiplatformnú desktopovú aplikáciu a web aplikáciu. Výsledkom tejto práce je funkčná klientská aplikácia pre zobrazenie dát a spravovanie senzorického informačného systému Sensorical. V prvej kapitole je opísaný systém Sensorical, jeho dátový model a komunikácia so serverom. Druhá kapitola sa venuje analýze požiadaviek a štruktúre aplikácie. V tretej kapitole sú popísané nástroje, ktoré boli použité pri vývoji aplikácie. Štvrtá kapitola pojednáva o implementácií a piata kapitola je venovaná buildovaniu aplikácie.
\bigskip

\noindent
\textbf{Kľúčové slová}:  desktopová aplikácia, web aplikácia, React, Electron, senzorický informačný systém
\bigskip


\bigskip
\noindent
\textbf{Abstract}

\noindent 
Zverbík, Adam: Client application for data visualization as a part of sensory information system. [Bachelor thesis]. – Slovak University of Technology. Faculty of Materials Science and Technology; Institute of Applied Informatics, Automation and Mechatronics. – Thesis supervisor: Ing. Juraj Ďuďák, PhD. – MTF STU, 2020, 57p.


\bigskip
The aim of the bachelor thesis is to develop and implement multiplatform desktop application and web application. The result of this thesis is functional client application, which shows and manages data of the sensoric information system Sensorical. Data model of sensorical system and communication with server is described in the first chaper. The second chapter is dedicated to analyze requirements and structure of the application. The tools used in development process are described in the third chapter. The fourth chapter is about implementation and the fifth chapter describes building of the final application.

\bigskip
\noindent
\textbf{Key words: } desktop application, web application, React, Electron, sensory information system. 





\pagenumbering{arabic}
\pagestyle{fancy}
\setcounter{page}{7}

%\addcontentsline{toc}{chapter}{Obsah}
\setcounter{tocdepth}{3}
\tableofcontents
\cleardoublepage
% vynechany zoznam obrazkov a tabuliek
%\newpage
\cleardoublepage
\listoffigures
\addcontentsline{toc}{chapter}{Zoznam obrázkov}
\listoftables
\addcontentsline{toc}{chapter}{Zoznam tabuliek}
\cleardoublepage
%\pagestyle{plain}
\addcontentsline{toc}{chapter}{Zoznam skratiek}
\vspace{0pt plus 2cm}
\chapter*{Zoznam skratiek}
\begin{acronym}
\acro{SKRATKA }{ VYZNAM SKRATKY} 
\end{acronym}

%pouzitie v texte 
% \ac{SKRATKA} - long description (SKRATKA) 
% \ac{SKRATKA} - short - because it is second use 
% \acs{SKRATKA} - short  - this will force short form



\chapter*{Úvod}
\addcontentsline{toc}{chapter}{Úvod}

Internet. Dnes si život bez neho vieme iba ťažko predstaviť. Obrovské množstvo ľudí sa pripojí na internet každý jeden deň. Či už si chcú nájsť nejakú informáciu, objednať niečo z internetového obchodu, pozrieť seriál, film, alebo chcú komunikovať s ďalšími ľuďmi, pripojenými na druhom konci sveta. Internet je stále dostupnejší. Vďaka novým, vyspelejším technológiám sa stáva pripojenie stabilnejšie a rýchlejšie. Práve kvôli tomu sú internetové aplikácie stále populárnejšie. Takáto aplikácia je plnohodnotná aplikácia, ale používateľ si nemusí nič inštalovať do svojho počítača, jednoducho sa pripojí na internet a v pohodlí internetového prehliadača si aplikáciu spustí.

Vzhľadom na neustály pokrok technológií sú stále výkonnejšie a dostupnejšie aj samotné počítače. S rastúcim výkonom počítačov rastie aj popularita jazyka JavaScript a jeho frameworkov, nakoľko moderné počítače si s ním hravo poradia. To umožňuje vývoj moderných, dynamických aplikácií s použitím internetových technológií. Takýmito frameworkami sú aj React a Electron. React umožňuje beh jedno-oknovej (single-page) aplikácie v internetovom prehliadači, Electron zase v okne operačného systému. Kombináciou týchto dvoch frameworkov vznikne multiplatformná desktopová aplikácia dostupná aj z internetového prehliadača.

Senzorický informačný systém Sensorical je systém, ktorý zaznamenáva a analyzuje namerané environmentálne dáta. Dáta sú zaznamenávané meracími zariadeniami, ktoré merajú hodnoty v určitých časových intervaloch a po zaznamenaní tejto hodnoty komunikujú so vzdialeným serverom, kde sa tieto dáta ukladajú. Tento informačný systém obsahuje desktopovú aplikáciu, internetovú aplikáciu a aplikáciu pre operačný systém Android.  

Cieľom práce je naprogramovať a implementovať multiplatformnú desktopovú aplikáciu, ktorá dostala názov eAurela. Aplikácia eAurela bude dostupná aj z internetového prehliadača, čo značne zjednoduší spravovanie celého systému Sensorical. Aplikácia bude zobrazovať posledné namerané hodnoty z meracích zariadení a grafické výstupy z vybraných časových intervalov. Administrátori informačného systému budú mať možnosť z aplikácie spravovať celý systém.

\begin{comment}

\begin{code}
\begin{minted}{js}

\end{minted}
\end{code}
\end{comment}
\titleformat{\chapter}[display]{\normalfont\bfseries}{}{10pt}{\huge\thechapter \quad#1}

 %Pripadne doplniť dalsie
%\include{kapitoly/ciele}
\chapter{Webové služby}
Pojem webové služby (Web Services) sa začína vo väčšom množstve objavovať okolo roku 2000, kedy na nich firma Microsoft založila svoju, dnes veľmi používanú architektúru \textit{.NET}, IBM ich označila za revolúciu v e-business, SUN hovorí o novej generácií distribuovaných systémov a mohli by sme pokračovať \cite{WebServicesIntro}.

Webové služby sú významným krokom vo vývoji distribuovaných systémov. Nadväzujú na technológie \textbf{RPC} (Remote procedure calls - technológie vzdialeného volania procedúr), \textbf{DCOM} (Distributed Component Object Model - technológie rozširujúce možnosti COM), \textbf{CORBA} (Common Object Request Broker Architecture) a \textbf{RMI} (Remote Method Invocation - technológie vzdialeného volania metód objektov v jazyku \textit{Java}). Všetky tieto technológie umožňujú volať funckiu na vzdialenom systéme \cite{WebTechnologies}. 

Tieto technológie sú tzv. softvérové služby. Softvérová služba je niečo, čo prijme digitálnu (číslicovú) požiadavku a vráti digitálnu odpoveď. Podľa tejto definície, sú funkcia v jazyku C, objekt v Jave, procedúra uložená v SQL príklady softvérovej služby. Kedysi boli softvérové služby limitované konkrétnym jazykom, alebo platformou. Taktiež väčšinou neboli dostupné skrz sieť. Práve toto menia webové služby, ktoré sú jazykovo aj platformovo nezávislé. Túto nezávislosť zabezpečuje použitie formátu XML, ktorý slúži pre vzájomnú výmenu dát \cite{WebServices}.

\section{Aplikačné programové rozhranie}

V dnešnej dobe je zaužívaný pojem API. API je skratka pre \textit{Application Programming Interface} (Aplikačné programové rozhranie). Vo všeobecnosti je to čokoľvek, čo umožňuje jednotlivým častiam softvéru komunikovať medzi sebou. Jedná sa o súhrn funkcií, procedúr, tried, komunikačných protokolov a ďalších nástrojov. Je dôležité, aby metódy určené pre komunikáciu boli presne definované. Existuje viac druhov API, napríklad grafické API, API pre frameworky a knižnice, API operačných systémov a webové API.

\subsection{Webové API}

Webové API definuje, ako spolu komunikujú nejaké komponenty skrz internet, napríklad 2 časti tej istej aplikácie (web stránka si dáta dopytuje zo serveru pomocou požiadaviek), alebo 2 rôzne aplikácie (mobilná aplikácia sťahuje dáta zo serveru). Webové API sú vlastne synonymom pre webové služby, no kvôli zaužívanosti budem ďalej používať pojem API.

\subsubsection*{Druhy webových API}

Ako to už v informatike býva, existuje viacero spôsobov komunikácie. Sú to:
\begin{itemize}
\item Jednoduché API
\item API používajúce protokol \textbf{SOAP}
\item API používajúce architektúru \textbf{REST}
\item Graph API od Facebooku
\end{itemize}

\subsubsection*{Jednoduché API}
Jednoduché API zvyčanjne ponúkajú iba nejaký zoznam dát k stiahnutiu, napríklad počasie podľa mesta, ale neumožňujú zložitejšiú manipuláciu s dátami. Príklad výstupu jednoduchého API, ktoré je populárne medzi českými e-shopmi, ktoré beží na stránkach Českej národnej banky na adrese \url{https://www.cnb.cz/cs/financni_trhy/devizovy_trh/kurzy_devizoveho_trhu/denni_kurz.txt} je možné vidieť na obrázku \ref{fig:SimpleAPI}.

\begin{figure}[H]
\includegraphics[width=0.5\textwidth]{obrazky/Simple_WebService.png}
    \centering
    \caption{API vo formáte CSV zo stránky Českej národnej banky zo dňa 04.12.2020}
    \label{fig:SimpleAPI}
\end{figure}

\subsubsection*{SOAP} \label{soapRef}
Skratka SOAP znamená \textit{Simple Object Access Protocol}. Správy posielané pomocou protokolu SOAP sú obvykle založené na XML (značkovací jazyk, ktorý je podobný jazyku HTML). Oproti REST je SOAP skôr procedurálny (REST sa orientuje na dáta). To sa prejavuje aj v spôsobe volania - URL adresa pri používaní SOAP bude typicky obsahovať nejaké sloveso, na rozdiel od REST, kde bude typicky podstatné meno.

\subsubsection*{REST}
REST je v súčasnej dobe veľmi populárna architektúra rozhrania. Je to skratka \textit{REpresentational State Transfer}. Tento pojem zaviedol vo svojej dizertačnej práci Roy Fielding. Roy Fielding je jeden zo spolautorov protokolu HTTP, a teda REST tento protokol používa. REST implementuje základné štyri CRUD operácie. Tieto operácie sú Create, Read, Update, Delete. V protokole HTTP im odpovedajú tieto metódy:

\begin{itemize}
    \item GET
    \item POST
    \item PUT
    \item DELETE
\end{itemize}

Vďaka týmto 4 metódam je REST veľmi jednoduché na pochopenie a aj používanie. Oproti SOAP je stručnejšie a efektívnejšie. Aj napriek stručnosti obsahuje každá požiadavka všetky potrebné informácie potrebné k jej vybaveniu a server teda nemusí držať žiadny stav (je stateless). Z toho vyplýva, že pokiaľ aplikácia potrebuje držať nejaký stav, musí ho uchovávať klient. API, ktoré používa rozhranie REST, sa označuje ako \textit{RESTful API}.

\subsubsection*{Graph API}
Graph API vytvoril a zpopularizoval Facebook, ktorý cez neho prezentuje veľmi rozmanité dáta. Facebooku sa totiž môžeme spýtať toľko vecí, že by REST a dokonca aj SOAP požiadavky boli veľmi neprehľadné. Graph API používa pre reprezentáciu informácií koncept sociálnych grafov s vrcholmi a hranami. Vrcholy sú objekty, ako napríklad používateľ, fotka, komentár atď. Hrany sú spojenia medzi jednotlivými objektami, ako napríklad komentáre pod konkrétnou fotkou. Dáta sú uložené v poliach objektu \cite{WebAPI}.







\chapter{SOAP a XML}
Ako bolo spomenuté v časti \textit{\ref{soapRef}}, pojem SOAP je skratka pre \textit{Simple Object Access Protocol}, čo v preklade znamená \textit{jednoduhý protokol pre prístup  objektom}. Aplikáciam umožňuje komunikovať medzi sebou za použitia prevažne \textit{HTTP} a \textit{XML}. V zásade predstavuje paradigmu jednosmernej výmeny správ medzi jednotlivými uzlami bez stavu (stateless). Kombináciou jednosmerných výmen s funkciami poskytovanými základným transportným protokolom a (alebo) špecifickými informáciami aplikácie, možno SOAP použiť na vytvorenie zložitejších interakcií, ako je požiadavka - odpoveď, požiadavka - viacnásobná odpoveď atď. \cite{SoapRestComparison}.
\section{Vznik protokolu SOAP}
Už na začiatku, ako vzniklo WWW, bolo možné na webservere zavolať program a predať mu textové parametre vďaka URL adrese. Jednoducho sa na koniec URL adresy pridal \textit{?} a zaň sa uviedli názvy parametrov a ich hodnoty, oddelené znakom \textit{\&}. Keďže je ale URL adresa limitovaná dĺžkou, musel sa vymyslieť iný prístup. Bola vymyslená metóda \textit{POST} protokolu \textit{HTTP}, ktorá parametre predáva v tele \textit{HTTP} požiadavku. Metódou \textit{POST} je možné posielať akékoľvek dáta akejkoľvek dĺžky. Štandardizovaný bol ale typ nazvaný \textit{application/x-www-form-urlencoded}, ktorého tvar je zhodný s tvarom parametrov predávaných v URL adrese.

Neskôr začali prehliadače podporovať aj typ \textit{multipart/form-data}, ktorý umožňuje k textovým parametrom pridať obsah súboru.

S príchodom jazyka \textit{XML} bolo iba otázkou času, než niekoho napadlo posielať si metódou \textit{POST} dáta v \textit{XML}. \textit{XML} umožňuje zapísať lubovoľne zložité štruktúrované dáta do textového súboru platformovo nezávislým zbôsobom. Výhoda je, že sa predávané dáta nemusia obmedzovať na text, ale je možné predávať si zložité objekty a aj kolekcie objektov \cite{WebServicesIntro}.

\section{Popis protokolu SOAP}
Ako bolo spomenuté vyššie, protokol SOAP je flexibilný, nezávislý na platforme, v ktorom sa komunikujúce strany považujú za rovnocenné. Delia sa iba podľa príznaku \textit{klient-server}. Skladá sa z niekoľkých častí:
\begin{itemize}
\item prvá časť: obálka (envelope):
\begin{itemize}
\item popisuje obsah správy,
\item obsahuje niekoľko parametrov na vysvetlenie toho, ako sa má príjemca správať pri spracúvaní obálky,
\end{itemize}
\item druhá časť: obsahuje pravidlá, ako sa majú kódovať jednotlivé inštancie dátových typov,
\item posledná časť:
\begin{itemize}
\item popis nasadenia obálky,
\item pravidlá kódovania dátových typov na reprezentáciu RPC (Remote procedure call) volaní a odpovedí využívajúc protokol HTTP.
\end{itemize}
\end{itemize}

SOAP ale nie je odkázaný iba na HTTP. Môže použiť hociaký transportný protokol (napríklad SMTP a i.), stačí, že ho podporuje implementácia SOAP. Dnes už existuje mnoho implementácií SOAP, napríklad to sú \textbf{SOAP:Lite} pre \textit{PERL}, \textbf{Apache Axis} a \textbf{Apache SOAP 2} pre \textit{Java}, tiež existujú implementácie pre jazyk \textit{C\#}, \textit{Delphi}, \textit{Visual Basic} atď \cite{Implementations}.
\begin{figure}[H]
\includegraphics[width=1\textwidth]{obrazky/SOAP_preview.png}
    \centering
    \caption{SOAP správa s použitím HTTP protokolu \cite{Description}}
    \label{fig:SOAP-HTTP}
\end{figure}

Ako je možné vidieť na obrázku \ref{fig:SOAP-HTTP}, prvé 4 riadky sú štandartné parametre protokolu HTTP. Používa sa HTTP metóda \textit{POST} a využíva sa HTTP verzia 1.0 (prvý riadok), volá sa host www.mindstrm.com a používané kódovanie je XML (Content-Type). Content-Length je informácia o dĺžke správy. Piaty riadok obsahuje parameter \textit{SOAPAction}, ktorý už je špecifický pre protokol SOAP. Ten príjemcovi hovorí, čo je obsahom správy. Tento parameter je zavedený iba za účelom šetrenia času, čiže ak príjemca po skontrolovaní parametru SOAPAction zistí, že túto požiadavku nemôže spracovať, nemusí ju zbytočne parsovať. Samotná SOAP požiadavka obsahuje tagy \textit{Envelope}, \textit{Header} (tento tag v obrázku nie je uvedený) a \textit{Body}.

\subsubsection*{Envelope}
Envelope je párový tag, ktorý zabaľuje celú správu. Tento tag obsahuje \textit{Namespace-y} prislúchajúce protokolu SOAP. Pre SOAP verzie 1.1 je to \url{http://schemas.xmlsoap.org/soap/envelope}. Pre SOAP verzie 1.2 zase \url{http://www.w3.org/2001/12/soap-envelope}. Zvyknú sa tu ešte definovať Namespace-y pre xml-schému a xml-encoding.

\subsubsection*{Header}
Je voliteľný tag, ktorý definuje správanie príjemcu. V prípade, že \textit{Header} tag nie je vynechaný, je to prvý priamy potomok tagu Envelope. Zvyčajne sa používa pre definovanie parametrov pre prihlásenie, transakcie, a podobne. Potomkovia tagu \textit{Header} môžu použiť 2 parametre: \textit{Actor} a \textit{MustUnderstand}.
\begin{itemize}
\item Actor definuje príjemcu správy. Ak je príjemca iný, nesmie správu použiť a musí zahodiť príslušný Header subelement. Používa sa to na komunikáciu s nejakou medzistanicou.
\item MustUnderstand hovorí, že ak v priebehu spracovania sa nepodarilo pochopiť ľubovoľný z prvkov, tak sa musí odpovedať špecifickou chybou pre SOAP, tzv. \textit{SOAP fault}.
\end{itemize}

\subsubsection*{Body}
Body tag obsahuje informáciu o volaných rozhraniach. Jednotlivé subelementy volajú príslušné metódy a vnútri obsahujú príslušné parametre volania s ich udaným typom (v prípade, že to je požadované) \cite{Description}.



\chapter{Nástroje použité na vývoj}
Vo vývoji aplikácie eAurela boli použité viaceré nástroje. Keďže má byť aplikácia eAurela dostupná na platformách Windows, Linux, Mac OS a aj ako internetová aplikácia (nefunkčná požiadavka R01), je naprogramovaná pomocou dvoch javascriptových frameworkov: \textit{React.js} a \textit{Electron.js}. Keďže je Electron postavený na \textit{Node.js}, bolo počas vývoja potrebné použiť aj toto javascriptové behové prostredie.

\section{Framework React.js} \label{sec:IntroToReact}
React je javascriptová knižnica s otvoreným zdrojovým kódom (open-source) od spoločnosti Facebook. Slúži na tvorbu používateľského rozhrania (UI). Od rôznych iných javascriptových frameworkov sa React odlišuje tým, že sa sústredí iba na jednu špecifickú oblasť a tvorí iba vrstvu pohľadu (view), ktorá prezentuje dáta používateľovi.
Jednou z najväčších výhod Reactu je, že povoluje JSX syntax. Táto syntax je rozšírením javascriptu a umožňuje nám písať HTML kód priamo do javascriptového kódu, čo zjednodušuje celý proces \cite{JSX}.
Základným stavebným prvkom React aplikácie sú tzv. komponenty (components). Komponenty sú rôzne znovu použiteľné JSX elementy so zapuzdrenou funkcionalitou. Ich skladaním vzniká komplexná UI aplikácia. Tieto komponenty majú svoje vlastnosti (props) a spravujú svoj vnútorný stav (state). Tento spôsob práce s dátami vedie k viac predvídateľnému správaniu a aj ľahšiemu ladeniu. React je možné používať aj s ďalšími podobne zameranými knižnicami, ako je napríklad Redux, alebo Electron \cite{React}.
\subsection{Komponenty a vlastnosti}
Komponenty nám umožňujú rozdeliť používateľské rozhranie na nezávislé, znovu použiteľné časti. Koncepčne sú ako javascriptové funckie. Prijímajú ľubovoľné vstupy (props) a vracajú reactové elementy, popisujúce, čo by sa malo zobraziť na obrazovke. Komponenty môžu byť funkcionálne alebo triedne.
\subsubsection*{Funkcionálne komponenty}
Najjednoduchší spôsob, ako zadefinovať komponent, je napísať javascriptovú funkciu (viď kód \ref{functionalComponent}).
\begin{lstlisting}[caption={Príklad funkcionálneho komponentu \cite{ReactComponentsAndProps}}, label={functionalComponent}]
function Welcome(props) {
  return <h1>Hello, {props.name}</h1>;
}
\end{lstlisting}

Táto funckia je plnohodnotný komponent, pretože prijíma vlastnosť (props ako properties) s dátami a vracia reactový element.
\subsubsection*{Triedny komponent}
Pre zadefinovanie komponentu sa môže taktiež použiť trieda z ES6 (verzia javascriptu ECMAScript 6).
\begin{lstlisting}[caption={Príklad triedneho komponentu \cite{ReactComponentsAndProps}}, label={classComponent}]
class Welcome extends React.Component {
  render() {
    return <h1>Hello, {this.props.name}</h1>;
  }
}
\end{lstlisting}
Takto zadefinovaný komponent je rovnaký ako uvedený funkcionálny komponent v kóde \ref{functionalComponent}.
\subsubsection*{Vykreslenie komponentu}
Spomenuté elementy, popisujúce čo by sa malo na obrazovke vykresliť môžu byť buď reactové elementy, ktoré reprezentujú DOM (Document Object Model) značky (tagy). Každopádne elementy môžu reprezentovať aj používateľom zadefinované komponenty \cite{ReactComponentsAndProps}. 
\begin{lstlisting}[caption={Dom tag a používateľom definovaný komponent} \cite{ReactComponentsAndProps}, label={elements}]
// DOM tag
const element = <div />; 
// používateľom definovaný komponent
const element = <Welcome name="Sara" />; 
\end{lstlisting}
Keď React uvidí element reprezentujúci používateľom zadefinovaný komponent, podá tomuto komponentu JSX atribúty a všetkých potomkov ako jeden objekt. Tento objekt sa nazýva \textit{props} (vlastnosti). React sa ku komponentom začínajúcim malým písmenom správa ako k DOM značkám (napríklad <div /> reprezentuje HTML div tag). Používateľom definovaný komponent musí začínať veľkým písmenom. Je to z toho dôvodu, že React predáva tieto elementy metóde \textit{React.createElement(parameter)}. V prípade, že sa jedná o element začínajúci malým písmenom, tejto metóde sa ako parameter podá názov elementu ako textový reťazec. Ak element začína veľkým písmenom, tejto metóde sa podá celý komponent ako objekt \cite{ReactJSXinDepth}. 
\subsubsection*{Vlastnosti v komponentoch}
Konfiguráciu a predávanie dát do komponentu majú na starosti vlastnosti (props). Vlastnosti sú dáta určené iba na čítanie (read-only). Či je komponent definovaný ako trieda alebo funkcia, nikdy nesmie modifikovať a meniť hodnotu vlastnostiam. Funkcie, ktoré nikdy neprepisujú svoje vstupné parametre sa nazývajú čisté (pure) funkcie. Takáto funkcia vráti pri rovnakom vstupe vždy ten istý výsledok. Opakom takejto funkcie je funkcia nečistá (impure), ktorá môže meniť svoj vstup. Všetky reactové komponenty sa musia správať ako čisté funkcie vzhľadom na ich vlastnosti \cite{ReactComponentsAndProps}.
\subsection{Stav aplikácie}
Stav (state) je podobný vlastnostiam. Používa sa ale výhradne vo vnútri komponentu pre riadenie toku dát. Sú to teda privátne dáta komponentu. Stav sa deklaruje v konštruktore triedneho komponentu. V konštruktore sa nastaví počiatočný stav, ktorý môže byť počas behu aplikácie menený. Pre zmenu stavu sa používa metóda \textit{setState()}, a teda stav nikdy nie je menený priamo.
\begin{lstlisting}[caption={Aktualizovanie stavu} \cite{ReactStateLifecycle}, label={setState}]
// Wrong
this.state.comment = 'Hello';
// Correct
this.setState({ comment: 'Hello' });
\end{lstlisting}
Výraz \textit{this.state} môže byť pre nastavenie stavu použitý iba v konštruktore komponentu. Aktualizovanie stavu spolu s vlastnosťami môže byť asynchrónne, a preto by sa nemalo spoliehať na ich hodnoty pre výpočet nového stavu. Aby sa zabránilo tomuto správaniu, používa sa druhý spôsob zápisu metódy \textit{setState}, ktorý prijíma funkciu a nie objekt. Táto funkcia dostane ako prvý parameter predchádzajúci stav a ako druhý parameter dostane vlastnosť, v čase kedy je vykonaná aktualizácia. V kóde \ref{setStateFunction} je použitá šípková funkcia, každopádne to tiež správne funguje s regulárnou funkciou \cite{ReactStateLifecycle}.
\begin{lstlisting}[caption={Použitie šípkovej funkcie vo vnútri setState \cite{ReactStateLifecycle}}, label={setStateFunction}]
this.setState((state, props) => ({
  counter: state.counter + props.increment
}));
\end{lstlisting}

\section{Framework Electron.js}
Electron je open-source projekt, ktorý slúži na vývoj a implementáciu multiplatformných desktopových aplikácií. Aplikácie naprogramované pomocou Electronu sú kompatibilné s operačnými systémami Mac, Windows a Linux.

Electron používa Chromium, čo je open-source verzia prehliadača Google Chrome. Ďalej používa Node.js, čo umožňuje vyvíjať desktopové aplikácie pomocou HTML, CSS a javascriptu. V tomto frameworku sú naprogramované populárne aplikácie, ako napríklad Visual Studio Code, Atom editor, Discord a mnoho ďalších.

\subsection{Vývoj Electron aplikácie}

Pokiaľ ide o vývoj, Electron aplikácia je v podstate Node.js aplikácia. Východiskovým súborom je \textit{package.json}, ktorý je totožný s východiskovým modulom Node.js. Súbor \textit{package.json} obsahuje informácie o hlavnom procese a taktiež všetky závislosti aplikácie.

\begin{lstlisting}[caption={Vzor súboru package.json},label={lst:package}]
{
  "name": "eaurela",
  "version": "0.1.0",
  "main": "./main.js",
  "private": true,
  "dependencies": {
    "react": "^16.12.0"
  },
  "scripts": {
    "start": "react-scripts start",
    "build": "react-scripts build",
    "electron-win": "set ELECTRON_DISABLE_SECURITY_WARNINGS=true && set ELECTRON_START_URL=http://localhost:3000/ && electron .",
    "dev-win": "concurrently -k \"npm start\" \"wait-on http://localhost:3000 && npm run electron-win\"",
    "package-win": "electron-packager . --platform=win32 --arch=x64"
   },
   },
  "devDependencies": {
    "concurrently": "^5.0.0",
    "electron": "^9.0.0",
    "wait-on": "^3.3.0"
  }
}
\end{lstlisting}
Súbor \textit{package.json} je vlastne javascriptový objekt, pričom jednotlivé vlastnosti musia byť v úvodzovkách. Skladá sa z viacerých častí. Prvok \textit{"name"} je názov aplikácie, \textit{"version"} hovorí o verzii aplikácie. Jedna z najdôležitejších vlastností je \textit{"main"}. Tá odkazuje na hlavný proces, ktorý sa spustí pri spustení aplikácie. Ďalšie dôležité prvky sú \textit{"dependencies"} a \textit{"scripts"}. Dependencies hovoria o nainštalovaných knižniciach a ich verziách. Scripts sú skripty dôležité napríklad pre spúšťanie, testovanie alebo buildovanie aplikácie (buildovanie aplikácie je popísané v časti \ref{build}). Skripty sa spúšťajú z príkazového riadku (respektíve terminálu) pomocou príkazu \textit{npm} (node package manager). Vlastnosť \textit{"devDependencies"} hovorí o nainštalovaných závislostiach dostupných iba vo fáze vývoja aplikácie.

\subsection{Procesy Electron aplikácie} \label{procesy}
Electron aplikácia má viacero procesov. Hlavný proces (main process) má na starosti prácu so súbormi, komunikáciu s operačným systémom. 

Vykreslovacie procesy (renderer processes) vytvárajú obsah aplikácie v oknách prehliadača.

V Electron aplikácií, súbor \textit{package.json} spustí hlavný skript nazývaný hlavný proces. V našom prípade sa hlavný proces nazýva \textit{main.js} a nachádza sa v rovnakom priečinku ako súbor \textit{package.json} (viď kód \ref{lst:package}, vlastnosť \textit{"main"}). Skript, ktorý beží v hlavnom procese môže zobrazovať grafické programové rozhranie (Graphical User Interface) vytváraním webových stránok. Electron aplikácia má vždy práve jeden hlavný proces.

Nakoľko Electron používa Chromium pre zobrazovanie webových stránok, jeho viac procesová architektúra je tiež použitá. Každá webová stránka v Electron aplikácií spúšťa jej vlastný proces. Tento proces sa nazýva vykresľovací proces.
\subsubsection*{Modul BrowserWindow a rozdiel medzi hlavným a vykresľovacím procesom}
Modul BrowserWindow nám umožňuje vytvoriť nové okno prehliadača, alebo spravovať už existujúce okno. Každé okno prehliadača je separátny proces - vykresľovací proces.

Hlavný proces vytvára webové stránky vytváraním \textit{BrowserWindow} inštancií. V každej inštancií \textit{BrowserWindow} beží webová stránka vo svojom vlastnom vykresľovacom procese.

Hlavný proces spravuje všetky webové stránky a ich korešpondujúce vykresľovacie procesy. Každý vykresľovací proces je izolovaný a stará sa iba o webovú stránku bežiacu v ňom \cite{ElectronProcesses}.

\begin{lstlisting}[caption={Vytvorenie inštancie BrowserWindow v hlavnom procese main.js},label={lst:BrowserWindow}]
const electron = require('electron');
const { app, BrowserWindow} = electron;
let mainWindow;
const createWindow = () =>
{
  mainWindow = new BrowserWindow({
    width: 1250,
    height: 750,
    title: "eAurela",
    resizable: true,
   });
app.on('ready', createWindow);
\end{lstlisting}
Hlavný proces \textit{main.js} by mal vytvárať všetky okná a mal by obsluhovať všetky systémové udalosti, ktoré sa v aplikácií môžu vyskytnúť, napríklad vytvorenie okna, ak je obsah aplikácie načítaný \cite{BrowserWindow} (viď kód \ref{lst:BrowserWindow}).









\chapter{Implementácia aplikácie eAurela}
Celková implementácia všetkých častí aplikácie eAurela vychádza zo špecifikácie požiadaviek a návrhu štruktúry. Sú tu popísané jednotlivé časti aplikácie, ich funkcionalita a implementácia. Prvá časť tejto kapitoly sa bude venovať používateľskému rozhraniu, následne bude vysvetlené, ako funguje navigácia v aplikácií. Ďalšia časť popisuje komunikáciu so serverom, získavanie a uchovávanie dát. Na záver je v kapitole popísaný dizajn aplikácie a zobrazenie dát pomocou dizajnových prvkov a pomocou grafov z knižnice \textit{recharts}.
\section{Používateľské rozhranie}
Všetky časti aplikácie, ktoré spolu vytvárajú používateľské rozhranie (UI), sú implementované v súlade s navrhnutou štruktúrou v časti \ref{structure}. Pomocou týchto častí používateľ pracuje a komunikuje s aplikáciou. Každá časť aplikácie je samostatný reactový komponent, ktorý sa môže skladať aj z viacerých komponentov. V princípe rozdeľujú tieto komponenty aplikáciu na tri časti:
\begin{itemize}
\item prístup k dátam,
\item prezentácia dát používateľovi,
\item spravovanie systému (iba pre administrátorov).
\end{itemize}
\subsubsection*{Prístup k dátam}
Pre prístup k dátam slúžia komponenty \textit{Home} a \textit{Appilist}. Komponent \textit{Home} je prvý vykreslený komponent po spustení aplikácie eAurela. Po prihlásení do systému sa zobrazí komponent \textit{Applist} (viď obr. \ref{fig:HomeUI} a obr. \ref{fig:ApplistUI}).
\begin{figure}[H]
\includegraphics[width=.8\textwidth]{obrazky/HomeUI.png}
    \centering
    \caption{Prihlasovacia obrazovka (komponent \textit{Home})}
    \label{fig:HomeUI}
\end{figure}
\begin{figure}[H]
\includegraphics[width=.8\textwidth]{obrazky/ApplistUI.png}
    \centering
    \caption{Obrazovka pre výber aplikácie (komponent \textit{Applist})}
    \label{fig:ApplistUI}
\end{figure}
\subsubsection*{Prezentácia dát používateľovi}
Po výbere aplikácie zo spoločnosti v komponente \textit{Applist} sa vykresľuje komponent \textit{Sectors}. V tomto komponente si používateľ môže prezerať poslednú nameranú hodnotu zo senzorov v danom sektore, alebo si môže zvoliť vykreslenie grafu, ktorý zobrazuje krivky zo všetkých senzorov za určité obdobie. Tieto dve časti komponentu \textit{Sectors} je možné vidieť na obrázkoch obr. \ref{fig:SectorsUI}. a obr. \ref{fig:GraphsUI}.
\begin{figure}[H]
\includegraphics[width=.8\textwidth]{obrazky/SectorsUI.png}
    \centering
    \caption{Posledné namerané hodnoty v oblasti \textit{Letné sídlo} (komponent \textit{Sectors})}
    \label{fig:SectorsUI}
\end{figure}
\begin{figure}[H]
\includegraphics[width=.8\textwidth]{obrazky/GraphsUI.png}
    \centering
    \caption{Vykreslenie grafu z oblasti \textit{Letné sídlo} (komponent \textit{Sectors})}
    \label{fig:GraphsUI}
\end{figure}
\subsubsection*{Spravovanie systému}
Z funkčých požiadaviek R02 a R04 vyplýva, že administrátori systému budú mať možnosť spravovať systém cez aplikáciu eAurela. V prípade, že je používateľ na serveri pre autorizáciu uvedený ako administrátor, v komponente \textit{Applist} po kliknutí na \textit{Možnosti} sa mu zobrazí možnosť prejsť do časti spravovania tohto serveru. Podobne aj v komponente \textit{Sectors} sa používateľovi zobrazí možnosť spravovania danej aplikácie v prípade, že má na serveri aplikácie uvedené administrátorské oprávnenia.
\begin{figure}[H]
\includegraphics[width=.8\textwidth]{obrazky/AuthAdminUI.png}
    \centering
    \caption{Spravovanie serveru pre autorizáciu (komponent \textit{AuthAdministration})}
    \label{fig:AuthAdminUI}
\end{figure}
\begin{figure}[H]
\includegraphics[width=.8\textwidth]{obrazky/ManageAdminUI.png}
    \centering
    \caption{Spravovanie konkrétnej aplikácie (komponent \textit{ManageAdministration})}
    \label{fig:ManageAdminUI}
\end{figure}
\section{Navigácia v aplikácií}\label{navigation}
Jadrom celej aplikácie je knižnica React Router (https://github.com/ReactTraining/react-router). Tá zabezpečuje navigáciu medzi oknami (komponentami) v aplikácii eAurela. Táto knižnica slúži pre vytvorenie navigácie v aplikáciách naprogramovaných vo frameworku React. V jedno-oknových aplikáciach existuje iba jedna HTML stránka. Táto stránka je znovu použitá pri vykreslení rôznych komponentov v závislosti od navigácie. V knižnici React router sú tri primárne kategórie komponentov:
\begin{itemize}
\item smerovače (routers), ako <BrowserRouter> a <HashRouter>,
\item porovnávače (route matchers), ako <Route> a <Switch>,
\item samotná navigácia (navigation), ako <Link>, <NavLink> a <Redirect>.
\end{itemize}
\subsubsection*{Smerovače}
Sada \textit{react-router-dom} sa používa v internetových projektoch a poskytuje komponenty \textit{BrowserRouter} a \textit{HashRouter}. V aplikácií eAurela je použitá práve sada react-router-dom. Hlavný rozdiel medzi komponentami \textit{BrowserRouter} a \textit{HashRouter} je spôsob ako pracujú s URL adresou a ako komunikujú s internetovým serverom (Nodejs serverom). \textit{BrowserRouter} používa regulárny tvar URL adresy (tak ako ho je možné vidieť v prehliadači). Tento spôsob práce s URL adresou by bol dostačujúci v prípade, že by bola aplikácia zobrazovaná iba v prehliadači. Vzhľadom na to, že aplikácia bude slúžiť aj ako desktopová aplikácia, bolo výhodnejšie použiť práve \textit{HashRouter}. Ten aktuálnu URL adresu ukladá ako mriežkovú časť (hash portion), teda URL adresa môže vyzerať napríklad \textbf{http://priklad.com/\#/hladana/stranka}. Nakoľko mriežka nie je nikdy posielaná na server, netreba server špeciálne konfigurovať a navigácia bude fungovať v internetovom prehliadači a aj v desktopovej aplikácií. Pre správne použitie smerovača, je potrebné, aby bol umiestnený v jadre komponentovej hierarchie.
\subsubsection*{Porovnávače}
Existujú dva komponenty pre porovnávanie: \textit{Switch} a \textit{Route}. Keď je vykreslený \textit{Switch}, prehľadá všetkých potomkov \textit{Route}, kým nenájde jedného, ktorého adresa sa zhoduje s aktuálnou URL adresou. Ak ho nájde, vykreslí ho a všetky ostatné komponenty ignoruje. Ak nenájde žiadnu zhodu, nevykreslí nič (null).
\subsubsection*{Navigácia}
React Router poskytuje \textit{Link} komponent na vytváranie odkazov v aplikácií. Kedykoľvek je vykreslený \textit{Link}, vykreslí sa vlastne HTML tag <a>. Komponent \textit{NavLink} je špeciálny druh \textit{Link} komponentu, ktorému sa môže nastaviť vlastnosť \textit{active} v prípade, že sa cesta zhoduje s URL adresou.
Kedykoľvek treba vynútiť navigáciu, môže sa použiť komponent \textit{Redirect}. Keď sa vykreslí, bude navigovať podľa jeho vlastnosti \textbf{to} \cite{react-router}. Použitie komponentu \textit{Redirect} je ukázané a popísané v kóde \ref{lst:ProtectedRoute}.
\begin{lstlisting}[caption={Zjednodušená ukážka použitia react-router-dom komponentov v aplikácií}, label={lst:Routing}]
<HashRouter>
    <div className="App">
        <Switch>
            <Route exact path={"/"} component={Home} />
            <ProtectedRoute exact path={"/applist"}
                component={AppList}
             />  
            <ProtectedAppRoute exact path={"/sectors"}
                component={Sectors}
            />
        </Switch>
     </div>
</HashRouter>
\end{lstlisting}
\subsubsection*{Vytvorenie chránenej cesty}
Ak nastane situácia, kedy je ku komponentu možné pristúpiť iba v určitom prípade (napríklad používateľ musí byť prihlásený, aby získal prístup k určitým častiam aplikácie), treba použiť tzv. chránenú cestu (\textit{ProtectedRoute}). Keďže React Router taký komponent neposkytuje a do aplikácie majú prístup iba prihlásený používatelia, je potrebné chránenú cestu vytvoriť. Chránená cesta sa vytvorí modifikovaním už existujúceho komponentu \textit{Route}. Vytváranie chránenej cesty je ukázané v kóde \ref{lst:ProtectedRoute}.

\begin{lstlisting}[caption={Vytvorenie chránenej cesty z komponentu <Route>}, label={lst:ProtectedRoute}]
import React from 'react';
import { Route, Redirect } from 'react-router-dom';
export const ProtectedRoute = ({ component: Component,
                                            path, ...rest }) => {
    // v premennej sa uchováva stav, či je používateľ prihlásený
    const isAuth = sessionStorage.getItem('isLoggedIn'); 
    return (
        <Route    // vykreslenie klasického komponentu <Route>
            path={path}
            {...rest}
            render={props => {
                return isAuth ?
                    <Component {...props} {...rest} /> 
    /* ak používateľ nie je prihlásený, vykreslí sa <Redirect>,
    ktorý používateľa presmeruje na obrazovku Home */
                    : <Redirect to="/" />
            }} />
    )
}
\end{lstlisting}

\section{Získavanie a uchovávanie dát zo servera}
Všetky dáta, ktoré aplikácia eAurela získava a prezentuje používateľovi, získava zo vzdialeného servera. Prístup k údajom na serveri zabezpečuje REST API (časť \ref{API}). Aplikácia neponúka ukladanie dát pre neskoršie zobrazenie, ale vždy zobrazuje aktuálne dáta (funkčné požiadavky R06 a R08).
\subsection{Získavanie dát zo servera}
Existuje viacero spôsobov, ako komunikovať so vzdialeným serverom v React aplikácií. Väčšina vyžaduje dodatočnú inštaláciu knižnice určenej pre túto komunikáciu. V aplikácii eAurela však bol použitý spôsob priamo dostupný vo frameworku React, bez nutnosti inštalovať dodatočnú knižnicu. Pre túto komunikáciu slúži metóda \textit{fetch}. Táto metóda preberá minimálne jeden parameter a to práve URL adresu servera. V aplikácii eAurela sa odovzdávajú tejto metóde vždy 2 parametre. Prvý parameter je textový reťazec a predstavuje URL adresu servera. Druhý parameter je objekt, ktorého povinná vlastnosť je metóda (method). Tu sa určí typ metódy (post, get, put alebo delete). Ďalej sa v tomto objekte môže zadefinovať hlavička (headers) a telo (body) požiadavku. Tieto vlastnosti nie sú povinné a zadávajú sa podľa toho, ako je to zadefinované na strane servera. Obsahujú informácie, ktoré sú nevyhnutné pre správne odoslanie požiadavku. V hlavičke to môžu byť napríklad prihlasovacie údaje používateľa, JWT token pre autorizovanie, alebo aj meno aplikácie, do ktorej sa používateľ snaží prihlásiť. Pri vytváraní alebo upravovaní používateľa, spoločnosti, senzora a podobne, sa v tele požiadavku odošle táto informácia ako JSON objekt konvertovaný na textový reťazec.
\begin{lstlisting}[caption={Príklad požiadavky pri vytváraní nového používateľa}, label={requestBody}]
body: JSON.stringify({
                    login: this.state.login,
                    password: sha256(this.state.password),
                    firstname: this.state.firstname,
                    surname: this.state.surname,
                    isAdmin: this.state.isAdmin,
                    enabled: this.state.enabled,
                    Companies: this.state.Companies
                })
\end{lstlisting}
\begin{lstlisting}[caption={Ukážka metódy fetch (zjednodušené prihlásenie používateľa do aplikácie)}, label={fetchRequest}]
fetch("http://login.nsoric.com/nsoric/auth/login/",
    {
        method: 'POST',
        headers: { login: name, password: sha256(password) }
    }
)
    .then(response => response.json())
    .then(result => {
        if (result.user.enabled) 
            this.props.handleSuccessfulAuth(result.user, result.JWT);
        else { 
            if (result.message === "user does not exists") 
                alert("Zle zadané prihlasovacie údaje!");
         }
\end{lstlisting}
Metóda fetch je asynchrónna a teda vracia dátový typ \textit{promise}. To znamená, že metóda vráti synchrónne objekt promise ako dočasnú návratovú hodnotu predtým, než bude známa reálna hodnota. Potom, čo sa asynchrónna operácia dokončí, objekt zavolá callbak s výsledkom alebo chybou. Je to z toho dôvodu, aby aplikácia mohla normálne fungovať, kým získa odpoveď zo servera. Promise môže nadobúdať 3 stavy:
\begin{itemize}
\item \textbf{Čakajúci (Pending)} - v tomto stave je zahájená asynchrónna operácia, ale ešte nie je známy výsledok,
\item \textbf{Splnený (Resolved alebo aj Fulfilled)} - v tomto stave je asynchrónna operácia dokončená úspešne a objekt volá úspešný callback,
\item \textbf{Zamietnutý (Rejected)} - v tomto stave je asynchónna operácia dokončená neúspešne a objekt volá neúspešný callback. \cite{Promise}.
\end{itemize}
Konkrétne príklady dátového typu promise z aplikácie je možné vidieť v časti \ref{serverResponses} na obrázku \ref{fig:response1} (splnený stav) a na obrázkoch \ref{fig:response404} a \ref{fig:response500} (zamietnutý stav). Na obrázku \ref{fig:response2} je vidieť hodnotu, ktorú vrátila callback funkcia po úspešnom dokončení asynchrónnej funkcie.

V kóde \ref{fetchRequest} sú v premennej \textit{response} uložené práve informácie o promise objekte (stav a hodnota). V premennej \textit{result} je už uložená finálna hodnota zo serveru vo formáte JSON.

\subsection{Uchovávanie dát}\label{subsec:propsandstate}
Ako bolo spomenuté v časti \ref{sec:IntroToReact}, komponenty v React aplikácií majú svoje vlastnosti (props) a spravujú svoj vnútorný stav (state). Na základe stavu a vlastností je postavená komunikácia medzi jednotlivými komponentmi v aplikácií. Vo vnútri jedného komponentu sa definuje stav a ten sa odovzdá druhému komponentu ako jeho vlastnosť. Problém nastáva pri obnovení stránky (okna aplikácie). Pri obnovení stránky sa obnovuje aj stav komponentov (a teda s tým spojené aj vlastnosti). Aby sa zabránilo strate dát uchovaných v stave komponentu, existujú 2 možnosti:
\begin{itemize}
\item ukladanie dát do lokálneho úložiska prehliadača (localStorage a sessionStorage),
\item nastavovanie stavu po každom vykreslení komponentu (metóda componentDidMount).
\end{itemize}
\subsubsection*{Lokálne úložisko prehliadača}
Internetové aplikácie môžu ukladať dáta lokálne do úložiska používateľovho prehliadača. Tieto úložiská sú takmer rovnaké, jediným rozdielom je, že dáta uložené v localStorage nemajú žiadnu expiračnú dobu. V aplikácií eAurela je používaný sessionStorage, ktorý je automaticky vymazaný pri zatvorení prehliadača (v tomto prípade aj pri zatvorení Electron aplikácie). Ten v aplikácii uchováva dôležité informácie zo servera, ktoré nemôžu byť počas chodu aplikácie stratené. Jedná sa napríklad o údaje prihláseného používateľa (aplikácia tak vie, kto je prihlásený), JWT token, ktorý je posielaný v každej požiadavke na server. Pri strate týchto údajov po obnovení stránky, by sa aplikácia stala nefunkčnou. Tieto úložiská uchovávajú dáta vo forme textového reťazca. Preto dáta iného dátového typu treba vždy pred uložením konvertovať na textový reťazec.
\begin{lstlisting}[caption={Príklad uloženia hodnôt do sessionStorage}, label={sessionStorage.setItem}]
sessionStorage.setItem('user', JSON.stringify(res1));
sessionStorage.setItem('JWT', res2);
\end{lstlisting}
V kóde \ref{sessionStorage.setItem} sú prvé parametre ('user' a 'JWT') názov, pod ktorým budú dáta dostupné a druhé parametre sú samotné dáta, ktoré sa ukladajú. Získanie dát zo sessionStorage je ukázané v kóde \ref{sessionStorage.getItem}.
\begin{lstlisting}[caption={Príklad použitia hodnôt zo sessionStorage}, label={sessionStorage.getItem}]
const user = JSON.parse(sessionStorage.getItem('user'));
const jwt = sessionStorage.getItem('JWT');
\end{lstlisting}
Síce sa sessionStorage vymaže automaticky pri zatvorení aplikácie, treba ho manuálne vymazať v prípade, že sa používateľ odhlási zo systému, no aplikáciu eAurela nechá otvorenú. Pre manuálne vymazanie celého obsahu úložiska slúži metóda \textit{sessionStorage.clear()}.
\subsubsection*{Metóda \textit{componentDidMount}}
Ďalšou možnosťou, ako predísť strate údajov, je posielanie požiadaviek a následné nastavovanie stavu v metóde \textit{componentDidMount}. Táto metóda prebehne vždy, keď je komponent vykreslený. Posielanie požiadaviek na server (fetch) je doporučené robiť práve v tejto metóde.
\begin{lstlisting}[caption={Odoslanie požiadavku v metóde componentDidMount()}, label={componentDidMount}]
componentDidMount() {
    this.setState({ isLoading: true })
    const ID = this.props.activeSector.id;
    fetch(this.state.url + "/manage/sector/" + ID + "/last-values/",
        {
            method: 'GET',
            headers: {Authorization: "Bearer " + this.state.jwt}
        }
    )
        .then(response => response.json())
        .then(result => {
            // nastavenie stavu z odpovede zo servera
            this.setState({ Sensors: result, isLoading: false })
        })
        .catch(error => console.log(error))
}
\end{lstlisting}
Keďže namerané hodnoty zo senzorov majú byť vždy aktuálne, získavajú sa v tejto metóde. Komponent tak vždy po obnovení stránky má novú odpoveď zo servera s aktuálnymi nameranými hodnotami. Komponent obsahujúci metódu \textit{componentDidMount} sa vždy vykresľuje na dvakrát. Napríklad v komponente \textit{Sectors} sa najskôr vykreslia základné informácie o oblastiach a sektoroch a ikona načítavania, ktorá hovorí o tom, že aplikácia komunikuje so serverom (prebieha metóda componentDidMount).
\begin{figure}[H]
\includegraphics[width=1\textwidth]{obrazky/componentDidMount.png}
    \centering
    \caption{Vykreslenie komponentu bez dát zo servera}
    \label{fig:DidMount}
\end{figure}
Hneď ako aplikácia dostane odpoveď zo servera, ktorá obsahuje aktuálne údaje, na obrazovku sa dodatočne vykreslia tieto údaje bez toho, aby sa ostatné časti komponentu znovu vykresľovali. 
\begin{figure}[H]
\includegraphics[width=1\textwidth]{obrazky/componentDidMount1.png}
    \centering
    \caption{Vykreslenie komponentu s dátami zo servera}
    \label{fig:DidMount1}
\end{figure}
\section{Návrh dizajnu aplikácie}
Komponenty v aplikácií sú základným stavebným blokom používateľského rozhrania. Ich vzhľad tvorí veľkú časť toho, ako bude aplikácia nakoniec vyzerať, ako ju bude vidieť samotný používateľ. Možností, ako upravovať štýl a vzhľad komponentov, je viacero. V aplikácií sú používané dva spôsoby úpravy vzhľadu komponentov: pomocou štylizovaných komponentov a pomocou CSS.
\subsection{Štylizované komponenty}\label{reactstrapcomponents}
Tento spôsob úpravy dizajnu je v aplikácií využívaný najviac. Pre jednoduchosť bola použitá knižnica \textit{reactstrap} (http://reactstrap.github.io/), ktorá obsahuje už vytvorené komponenty s hotovým dizajnom. Tie sú jednoducho vložené do vnútra komponentu, kde majú byť použité. Reactstrap je knižnica obsahujúca komponenty z \textit{React Bootstrap 4}. Knižnica nie je závislá na jQuery alebo na Bootstrap javascript.
\subsubsection*{Komponenty pre interakciu s používateľom: <Input>, <Table> a <Form>}
Keďže React používa syntax JSX (časť \ref{sec:IntroToReact}), pre interakciu s používateľom boli použité komponenty Input a Form. Tieto komponenty fungujú na podobnom princípe, ako v jazyku HTML. Input slúži ako vstup, kam používateľ zadáva údaje (text, heslo a podobne). Form je formulár, v ktorom sa nachádza viacero vstupných polí a potvrdením tohto formulára sa odosiela požiadavka na server.
\begin{figure}[H]
\includegraphics[width=0.5\textwidth]{obrazky/LoginForm.png}
    \centering
    \caption{Formulár pre prihlásenie do aplikácie}
    \label{fig:LoginForm}
\end{figure}
V časti spravovania systému boli údaje pre ich sprehľadnenie vypísané na obrazovku do tabuľky. Pri upravovaní údajov zo servera bol použitý formulár, ktorý obsahoval tabuľku so vstupnými poľami (viď obrázok \ref{fig:Table}).
\begin{figure}[H]
\includegraphics[width=0.5\textwidth]{obrazky/Table.png}
    \centering
    \caption{Formulár s tabuľkou pre vytvorenie používateľa}
    \label{fig:Table}
\end{figure}
\subsubsection*{Komponenty pre zobrazenie informácií: <Card>, <ListGroup>}
Komponent Card (karta) v aplikácií zobrazuje informácie získané zo servera. Karta sa skladá z viacerých ďalších komponentov (potomkov), ktoré slúžia pre presné zadefinovanie miesta, kde bude údaj zobrazený. To slúži pre prehľadné zobrazenie dát. Prvý doplňujúci komponent karty je hlavička karty (CardHeader). V komponente Applist sa do hlavičky vypisuje názov spoločnosti, spolu aj s malým logom. Pri údajoch zo senzorov (obrazovka Sectors) sa do hlavičky vypisuje názov senzora a stav batérie (ak daný senzor batériu má). Ďalší dôležitý komponent pri vykresľovaní karty je jej telo (CardBody). V okne Applist je do tela karty vložený titulok (CardTitle). Ten zobrazuje doplňujúce informácie o spoločnosti, ako adresu a podobne. Najdôležitejší komponent v tele karty je ListGroup. Tento komponent slúži pre vypísanie zoznamu aplikácií v danej spoločnosti, z ktorej si používateľ môže vybrať a prihlásiť sa do nej. V časti \textit{Sectors} je telo karty použité na zobrazenie ID senzora, poslednej nameranej hodnoty zo senzora spolu s dátumom a časom, kedy bola táto hodnota nameraná.
\begin{figure}[H]
\includegraphics[width=1\textwidth]{obrazky/Cards.png}
    \centering
    \caption{Použitie karty pre výpis údajov}
    \label{fig:Cards}
\end{figure}
\subsubsection*{Komponenty pre navigáciu: <Navbar>, <Nav>, <NavLink>} 
Tieto komponenty slúžia pre jednoduchú navigáciu, či už medzi jednotlivými časťami aplikácie, alebo pri prechádzaní informácií. Za zmienku stojí v tejto časti komponent \textit{NavLink}. Komponent s takým istým názvom je aj v knižnici \textit{react-router-dom} (časť \ref{navigation}). Tieto komponenty vykonávajú v podstate to isté, rozdiel je v dizajne. Nevýhoda tohto komponentu je obnovenie obrazovky po kliknutí na tento odkaz, s čím môžu nastať problémy, ako boli popísané v časti \ref{subsec:propsandstate}. Aby sa odstránilo obnovovanie stránky pri kliknutí na odkaz, priradí sa mu funkcionalita komponentu \textit{Link} z knižnice \textit{react-router-dom}, ktorý presmeruje používateľa bez obnovenia stránky. Takto zapísaný \textit{NavLink} sa správa ako \textit{Link}, pričom mu zostane jeho dizajn. Takisto obsahuje vlastnosť \textit{active}, ktorá v aplikácií slúži na zvýraznenie časti, v ktorej sa používateľ nachádza (viď obr. \ref{fig:navbar}).
\begin{lstlisting}[caption={Príklad prepojenia komponentov <NavLink> a <Link>}, label={NavLink}]
<NavLink
    tag={Link}      // priradenie funkcionality komponentu <Link>
    to={props.path} // cesta, kam bude používateľ presmerovaný
    onClick={handleClick} // funkcia vykonaná po kliknutí
    >  
    {props.link}    // hodnotu vlastnosti vidí používateľ ako odkaz
</NavLink>
\end{lstlisting}
Komponenty \textit{Nav} a \textit{Navbar} slúžia ako rodičovské komponenty pre zoskupenie viacerých prvkov na navigáciu, alebo zobrazenie rôznych informácií. Rozdiel je v tom, že \textit{Navbar} má viacero možností nastavenia vzhľadu (od ktorých sa odvíja aj vzhľad "potomkov"), ale dá sa použiť iba horizontálne, pričom \textit{Nav} sa dá použiť aj vertikálne.
\begin{figure}[H]
\includegraphics[width=.6\textwidth]{obrazky/SectorsNavbar.png}
    \centering
    \caption{Komponenty <NavLink>, <Nav> a <Navbar>}
    \label{fig:navbar}
\end{figure}
\subsection{Úprava vzhľadu pomocou CSS}
V aplikácií bol taktiež na úpravu vzhľadu použitý jazyk CSS. Pre nastavenie základných vlastností aplikácie eAurela, ako štýl písma a podobne, bol použitý klasický spôsob vloženia odkazu na externý .css súbor do koreňového komponentu aplikácie. Taktiež aj pre dodatočnú úpravu niektorých komponentov z knižnice \textit{reactstrap} bol zvolený tento spôsob. React dovoľuje ale aj používanie CSS priamo vo vnútri komponentov. Tento spôsob bol použitý v aplikácií najmenej a bol použitý hlavne na prvky, ktorých vzhľad bolo treba dynamicky meniť (ako napríklad komponent \textit{NavLink} pri nastavení spomenutej vlastnosti \textit{active}).
\begin{lstlisting}[caption={Dynamické nastavenie štýlu komponentu <NavLink>}, label={NavLinkStyle}]
<NavLink
    tag={Link}
    /* štýl sa bude meniť vzhľadom na pravdivostnú
    hodnotu premennej props.Graphs */
    style={!props.Graphs ? {
    color: "black",
    backgroundColor: "white",
    boxShadow: "0px 0px 0px 1px lightgrey"
    } : style}
    to="sectors"
    active={!props.Graphs}
</NavLink>
\end{lstlisting}
\section{Vykreslovanie grafov}
Vzhľadom na funkčnú požiadavku R08 bolo potrebné v aplikácií eAurela zabezpečiť prácu s grafmi. Pre framework React existuje viacero knižníc pre prácu a zobrazenie grafov. V aplikácii eAurela je použitá knižnica \textit{recharts} (http://recharts.org/en-US/) pre jej bezplatnú dostupnosť a veľmi dobre vypracovanú dokumentáciu. Graf sa vykresľuje v samostatnom komponente \textit{SensorGraphs}, ktorý je následné vykreslený v komponente \textit{Sectors} po zvolení možnosti \textit{"Časový záznam"}. Na nasledujúcom obrázku je zobrazený graf v aplikácii eAurela s hodnotami z časového intervalu tridsať dní.
\begin{figure}[H]
\includegraphics[width=1\textwidth]{obrazky/SensorGraphs.png}
    \centering
    \caption{Vykreslenie grafu v komponente <Sectors>}
    \label{fig:Graph}
\end{figure}
Knžinica recharts ponúka veľké množstvo grafov, či už čiarové, stĺpcové, koláčové a podobne. Keďže má aplikácia vykresľovať zmenu nameraných hodnôt z časového intervalu všetkých senzorov nachádzajúcich sa v sektore, bol použitý základný čiarový graf (LineChart). Vzhľadom na to, že v jednom grafe sa bude nachádzať viacero kriviek (podľa počtu senzorov), z dokumentácie \textit{recharts} bol vybraný čiarový graf s názvom \textit{"LineChartHasMultiSeries"}. Aby boli jednotlivé krivky správne vykreslené, komponent LineChart musí prevzať dáta upravené do správneho formátu. 
\begin{lstlisting}[caption={Vzor formátu dát pre LineChart}, label={LineChartHasMultiSeries}]
const series = [
  {
    name: 'Senzor 1',
    data: [
        {
            date: "",
            value: 
        },
        { ... }, ...
    ],
  },
  { ... }, ...
];
\end{lstlisting}
Konštanta \textit{series} je pole, kde každý prvok poľa predstavuje objekt s dvoma vlastnosťami: \textit{name} a \textit{data}. Vlastnosť \textit{name} obsahuje názov senzora, ktorý bude vypísaný v legende spolu s odpovedajúcou farbou krivky. Vlastnosť \textit{data} obsahuje konkrétne namerané dáta. \textit{Data} je taktiež pole objektov, kde jednotlivé objekty obsahujú taktiež dve vlastnosti: \textit{date} (dátum, kedy bola hodnota nameraná) a \textit{value} (nameraná hodnota). Táto konštanta je zadefinovaná vo funkcii pre vykreslenie komponentu (render).
\begin{lstlisting}[caption={Zadefinovanie konštanty series}, label={renderSeries}]
render() {
    const series = this.getChartData(Sensors, Measurements);
    return ( ... )
}
\end{lstlisting}
V odpovedi zo servera dostaneme dve polia: \textit{sensors}, kde sú uložené všeobecné informácie o senzoroch a \textit{measurements}, kde sú namerané hodnoty za dané obdobie. Takáto odpoveď zo serveru vyzerá podobne ako na obrázku obr. \ref{fig:ServerValues}.
\begin{figure}[H]
\includegraphics[width=1\textwidth]{obrazky/ServerValues.png}
    \centering
    \caption{Odpoveď zo servera pri vykreslovaní grafu}
    \label{fig:ServerValues}
\end{figure}
O spracovanie údajov zo servera do správneho formátu sa v aplikácií starajú dve metódy. Prvá metóda je \textit{getChartData}, ktorá ako parametre preberá namerané údaje zo servera (measurements), a taktiež všetky údaje o senzoroch (sensors). Táto metóda prechádza pole všetkých senzorov a pre vyhovujúce senzory (nevyhovujúce sú v tomto prípade senzory snímajúce stav batérie) priradí názov senzora. 
\begin{lstlisting}[caption={Metóda getChartData}, label={getChartData}]
getChartData(sensors, measurements) {
        let chartData = [];
            for (let i = 0; i < sensors.length; i++) {
                if (isNaN(sensors[i].uid) || this.DecToBin(sensors[i].uid).substring(0, 8) !== "00000000") {
                    chartData[i] = {
                        name: sensors[i].name,
                        data: this.getData(sensors[i].uid, measurements)
                    }
                }
            }
    return chartData;
}
\end{lstlisting}
Pre zjednodušenie ukážky boli z kódu odstránené dodatočné podmienky ošetrujúce výnimočné stavy, ktoré môžu nastať pri komunikovaní so serverom. Práve v týchto podmienkach je použitý parameter \textit{measurements}. Taktiež je použitý ako vstupný parameter funkcie \textit{getData}, ktorá vyplní vlastnosť \textit{data} konkrétnymi hodnotami (dátumom, časom a konkrétnou hodnotou). Táto metóda preberá ako vstupný parameter aj ID senzora, aby bolo možné prideliť namerané hodnoty k správnemu senzoru. Táto metóda prechádza všetky namerané údaje a hľadá tie, ktoré boli namerané senzorom s príslušným ID. 
 \begin{lstlisting}[caption={Metóda getData}, label={getData}]
getData(id, measurements) {
    let data = [];
    let k = 0;
    for (let i = (measurements.length - 1); i >= 0; i--) {
        for (let j = 0; j < measurements[i].values.length; j++) {
            if (id === measurements[i].values[j].sid) {
                data[k] = {
                    date: measurements[i].date.substring(0, 10) + " " + measurements[i].date.substring(11, 19),
                    value: measurements[i].values[j].value !== null ? measurements[i].values[j].value : null
                }   
                break;
            }
        }
        k++;    
    }
    return data;
}
\end{lstlisting}
Aj v tejto ukážke boli pre zjednodušenie kódu vynechané podmienky ošetrujúce stavy, spôsobujúce chyby v aplikácií.

Samotný komponent \textit{LineChart} sa skladá z viacerých ďalších komponentov, ktoré upresňujú rozloženie jednotlivých prvkov grafu a umiestnenie údajov. \textit{ResponsiveContainer} definuje šírku a výšku grafu. Komponenty \textit{XAxis} a \textit{YAxis} určujú, ktoré dáta sú vykreslené na každej z osí, či sa jedná o textové, alebo číselné údaje. Komponent \textit{Legend} vykresľuje legendu ku grafu, \textit{Tooltip} zobrazí nápovedu o konkrétnom dátume (nápovedu používateľ zobrazí prejdením kurzora do časti grafu, kde si chce prezrieť konkrétne hodnoty). \textit{Line} je samotná krivka, ktorej sa definujú dáta, ktoré sa použijú na vykreslenie. Keďže v aplikácií sa vykresľuje viacero kriviek do jedného grafu, bola použitá javascriptová metóda \textit{map} pre prechádzanie konštanty \textit{series} a pre každý jej prvok sa vykreslí vlastná krivka. Kód \ref{LineChart} zobrazuje implementáciu grafu v aplikácií aj s popisom jednotlivých častí a výsledný graf je možné vidieť na obrázku \ref{fig:TooltipGraph}.
 \begin{lstlisting}[caption={Implementácia grafu LineChart}, label={LineChart}]
<ResponsiveContainer width="95%" height="90%">
     <LineChart>
    <CartesianGrid strokeDasharray="3 3" />
    <XAxis
        dataKey="date"      // dáta zobrazené na osi X
        type="category"     // textový údaj (dátum a čas)
        allowDuplicatedCategory={false} // údaj sa nesmie opakovať
        minTickGap={10} />  // medzera medzi údajmi v popise
    <YAxis dataKey="value" />
    <Tooltip />
    <Legend />
    {series.map((s, index) => (
        <Line
            key={index}     
            dataKey="value"     // kľúčová hodnota
            data={s.data}       // pôvod dát
            name={s.name}       // názov senzoru
            stroke={this.getRandomColor()}  // farba krivky
            dot={false} />
    ))}
    </LineChart>
</ResponsiveContainer>
\end{lstlisting}
Vzhľadom na to, že \textit{recharts} pri vykreslení viacerých kriviek do jedného grafu priradí všetkým základnú (rovnakú) farbu, bola pre lepšie rozlíšenie kriviek implementovaná funkcia \textit{getRandomColor}, ktorá každej krivke priradí náhodnú farbu.
\begin{figure}[H]
\includegraphics[width=1\textwidth]{obrazky/TooltipGraph.png}
    \centering
    \caption{Ukážka grafu s popisom jednotlivých komponentov}
    \label{fig:TooltipGraph}
\end{figure}

\chapter{Uvedenie aplikácie do produkcie}\label{build}
Aby bola aplikácia dostupná a spustiteľná pre bežného používateľa, je potrebné vykonať build aplikácie. Tento proces slúži pre transformáciu zdrojových kódov v spustiteľnú aplikáciu. Keďže na vývoj aplikácie boli použité dva frameworky, build sa musí vykonať pre každý framework zvlášť. V tejto kapitole sú opísané postupy a nástroje použité pri vytváraní spustiteľnej aplikácie pre jednotlivé platformy. Všetky uvedené zmeny v zdrojovom kóde sa týkajú súboru \textit{package.json}.

\section{Buildovanie reactovej aplikácie}
Keďže electron má zobrazovať reactovú časť aplikácie, je potrebné vykonať build reactovej časti ako prvý. React má zabudovaný skript pre vykonanie buildu, takže nie je potrebné pridať ďalšiu knižnicu (viď kód \ref{lst:package}). Najdôležitejšia vec pri vykonaní buildu reactovej aplikácie je nastavenie vlastnosti \textit{homepage}. Táto vlastnosť je kľúčová pre React. Vďaka nej presne vie, kde sa spustiteľná aplikácia nachádza. V prípade, že by sa táto vlastnosť neuviedla, nebolo by možné vidieť reactovú časť aplikácie vo vnútri electronovej aplikácie.
\begin{lstlisting}[caption={Definovanie vlastnosti \textit{homepage} v aplikácii eAurela}, label={homepage}]
"homepage": "./",
\end{lstlisting}
Po zadefinovaní vlastnosti \textit{homepage} je všetko pripravené na spustenie buildovacieho skriptu. Úspešné ukončenie skriptu je zobrazené na obrázku obr. \ref{fig:reactBuild}.
\begin{figure}[H]
\includegraphics[width=.9\textwidth]{obrazky/reactBuild.png}
    \centering
    \caption{Úspešné použitie buildovacieho skriptu pre React}
    \label{fig:reactBuild}
\end{figure}
Po úspešnom prebehnutí tohoto skriptu sa v adresári aplikácie eAurela vytvorí nový priečinok s názvom \textit{build}. Tento priečinok obsahuje všetky statické súbory aplikácie, ktoré sú pripravené na prácu s electronovou časťou aplikácie.

\section{Buildovanie electronovej aplikácie}
Po úspešnom prebehnutí buildu v reactovej časti treba vykonať aj build electronovej časti, čo bude mať za výsledok spustiteľnú aplikáciu. Nakoľko Electron neobsahuje žiadny skript určený na buildovanie, bolo nutné použiť externú knižnicu. Použitá bola knižnica \textit{electron-packager} (https://github.com/electron/electron-packager). Po nainštalovaní knižnice je potrebné pridať buildovacie skripty do súboru \textit{package.json} pre každý operačný systém.
\begin{lstlisting}[caption={Skripty pre použitie knižnice \textit{electron-packager}}, label={electron-packager}]
// buildovací skript pre Mac OS
"package-mac": "electron-packager . --platform=darwin --arch=x64",
// buildovací skript pre Windows
"package-win": "electron-packager . --platform=win32 --arch=x64",
// buildovací skript pre Linux
"package-lin": "electron-packager . --platform=linux --arch=x64"
\end{lstlisting}
Pre zjednodušenie kódu boli uvedené iba povinné parametre jednotlivých skriptov.Prvý parameter (\textit{electron-packager}) hovorí, aby sa spustil práve \textit{electron-packager}. Druhý parameter (\textit{.}) hovorí, kde sa nachádza zdrojový kód na zverejnenie. Nasledujúce dva parametre označujú práve cieľovú platformu (\textit{--platfrom}) a typ architektúry (\textit{--arch}). Ďalšie voliteľné parametre by mohli byť názov aplikácie (ak nie je uvedený, použije sa názov uvedený pod vlastnosťou \textit{name} v súbore \textit{package.json}), ikona aplikácie a podobne. Úspešný priebeh skriptu \textit{package-win} je vyobrazený na obrázku obr. \ref{fig:electronBuild}.
\begin{figure}[H]
\includegraphics[width=1\textwidth]{obrazky/electronBuild.png}
    \centering
    \caption{Buildovanie aplikácie eAurela pre platformu Windows}
    \label{fig:electronBuild}
\end{figure}
Po úspešnom ukončení skriptu sa opäť vytvorí nový priečinok, ktorý obsahuje spustiteľnú aplikáciu, ako aj zdrojový kód.
\include{kapitoly/kapitola5}

\titleformat{\chapter}[display]{\normalfont\bfseries}{}{10pt}{\huge#1}
\chapter*{Záver}
\addcontentsline{toc}{chapter}{Záver}


Výsledkom práce je softvérový produkt s názvom eAurela. Tento softvérový produkt slúži na zobrazenie environmentálnych dát a spravovanie senzorického informačného systému Sensorical.

Aplikácia eAurela je naprogramovaná pomocou moderných internetových technológií. Na vývoj boli použité javascriptové frameworky React.js a Electron.js. Aplikácia je dostupná na operačných systémoch Windows, Linux a Mac OS ako desktopová aplikácia. Po nasadení bude aplikácia dostupná aj ako internetová aplikácia. 

Celá aplikácia eAurela pracuje na princípe volania existujúceho API a je určená výhradne používateľom systému Sensorical. Používatelia majú po prihlásení možnosť zvoliť si aplikáciu, respektíve inštaláciu, z ktorej si chcú prezerať namerané senzorické dáta, alebo si administrátori môžu zvoliť aj možnosť spravovania systému. Po vybraní aplikácie si používatelia môžu prezerať buď poslednú dostupnú nameranú hodnotu, alebo si môžu prezrieť grafické vyobrazenie nameraných hodnôt za nimi vybrané obdobie. Zobrazované dáta sú vždy aktuálne, zo servera sa získavajú pri voľbe aplikácie, alebo pri zmenení formy zobrazenia. Administrátori systému môžu spravovať a upravovať údaje uložené na serveri, môžu meniť používateľom prístupové práva v systéme, poprípade heslo. Administrátori majú možnosť spravovať údaje aj v rámci konkrétnej aplikácie. Tam majú možnosť registrovať do systému nových používateľov, alebo z aplikácie nejakého odobrať. Taktiež môžu upravovať prístupové práva v rámci aplikácie, alebo môžu meniť informácie o senzoroch, ako napríklad názov senzora a podobne.

Vzhľadom na vyššie uvedené funkcionality aplikácie eAurela je možné tvrdiť, že aplikácia spĺňa požiadavky, ktoré jej boli určené. V budúcnosti bude aplikácia rozšírená o rôzne funkcie, napríklad budú doplnené funkcionality v časti spravovania systému, ktoré počas vývoja neboli dostupné na strane servera  ako napríklad zmenenie rozmiestnenia senzorov v oblasti, vytvorenie novej spoločnosti alebo aplikácie, novej oblasti, sektora alebo senzora. Taktiež, ak to bude vyžadované, bude možné rozšíriť podporu aplikácie eAurela pre ďalšie jazykové mutácie.

%\renewcommand{\thepage}{} %pridane pre necislovanie bibliografie
\renewcommand{\bibsection}{\chapter{Zoznam použitej literatúry}}
\addcontentsline{toc}{chapter}{Zoznam použitej literatúry}
%\nocite{*}  %sluzi pre debugovanie, zakomentovat

\printbibliography[title=Zoznam použitej literatúry]

\cleardoublepage
\chapter*{Zoznam príloh}
\addcontentsline{toc}{chapter}{Zoznam príloh}

\begin{description}
\item[Príloha A : ] Použivateľská príručka
\item[Príloha B : ] Programatorská príručka
\item[Príloha C : ] Obsah CD disku
\end{description}


\cleardoublepage
\pagenumbering{Roman}

\addcontentsline{toc}{chapter}{Prílohy}

\begin{titlepage}
    \begin{center}
		\vspace*{10cm}
		\Huge \textbf{Prílohy}     
    \end{center}
\end{titlepage}
\setcounter{page}{2}
\appendix
\cleardoublepage

\titleformat{\section}[display]{\normalfont}{}{10pt}{\huge#1}
\titleformat{\subsection}{\Large\bf}{\textcolor{black}{#1}}{0pt}{}

 %Pripadne doplniť dalsie
%\addcontentsline{toc}{section}{Príloha A: Použivateľská príručka}
\section*{\textbf{Príloha A}: Použivateľská príručka}



%\addcontentsline{toc}{section}{Príloha B: Programatorská príručka}
\section*{\textbf{Príloha B}: Programatorská príručka}

\subsection{Názov sekcie}

 text...  text...  text...  text...  text...  text...  text...  text...  text...  text...  text...  text...  text...  text...  text...  text...  text...  text...  text...  text...  text...  text...  text...  text...  text...  text...  text...  text...  text...  text...  text...  text...  text...  text...  text...  text...  text...  text...  text...  text...  text...  text...  text...  text...  text...  text...  text...  text...  text...  text...  text...  text...  text...  text...  text...  text...  text...  text...  text...  text... 



%\addcontentsline{toc}{section}{Príloha C: Obsah CD disku}
\section*{\textbf{Príloha C}: Obsah CD disku}

Práca obsahuje prílohu vo formáte CD. 

\begin{itemize}
    \item Text bakalárskej práce vo formáte PDF. 
    \item Zdrojové kódy.
    \item Inštalačné súbory .\begin{itemize}
        \item Java ..., 
        \item MySQL,
        \item ... . 
    \end{itemize}
\end{itemize}

\end{document}